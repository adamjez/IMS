\documentclass[11pt,a4paper]{article}
% rozmery stranky
\usepackage[left=1.5cm,text={18cm, 25cm},top=2.5cm]{geometry}
% cestina a fonty
\usepackage[czech]{babel}
\usepackage[utf8]{inputenc}
\usepackage[T1]{fontenc}
% dalsi balicky
\usepackage{graphicx}
\usepackage{enumitem}
\usepackage{indentfirst}
\usepackage{url}
\usepackage[bookmarksopen,colorlinks,plainpages=false,urlcolor=blue,
unicode,linkcolor=black]{hyperref}


\begin{document}

  \begin{titlepage}
    \begin{center}
      \Huge
      \textsc{Fakulta informačních technologií \\ Vysoké učení technické v~Brně}
      \vspace{100px}
      \begin{figure}[!h]
        \centering
        \includegraphics[height=5cm]{logo}
      \end{figure}
      \\[50mm]
      \LARGE{Studie účelnosti zbudování vodní cesty Dunaj-Odra-Labe \,--\, 
             zadání č. 2}
      \vfill
    \end{center}
    \Large{Roman Blanco (xblanc01) \hfill 7.12.2014 \\
           Adam Jež (xjezad00)}

  \end{titlepage}

  \section{Úvod}

    Cílem zadaného projektu bylo prostudovat zdroje zabývající se účelností
    zbudování vodní cesty Dunaj-Odra-Labe a podle zjištěných údajů navrhnout
    roční poptávku po přepravě mezi zvolenými uzly. Součástí zadání bylo také
    navrhnout a implementovat model dopravní cesty, včetně stavebních prvků.

    \subsection{Autoři}

      Autory projektu jsou Roman Blanco (xblanc01) a Adam Jež (xjezad00) \,--\, studenti 3. ročníku bakalářského studia na Fakultě
      Informačních technologii VUT v Brně. Žádné osoby jsme za účelem zisku
      informací nekontaktovali.
      
      Jako prioritní zdroj informacím nám velmi dobře posloužily volně dostupné materiály na internetové stránce projektu
      zabývajicího se problematikou koridoru Dunaj-Odra-Labe a také poskytnuta literatura od týchž autorů.

    \subsection{Ověřování validity modelu}

      Ověřování validity modelu probíhalo pomocí experimentů. Ověřovalo se, zda
      modelová situace je odpovídající reálné situaci, přičemž informace byly
      čerpány pouze z věrohodných zdrojů. Jelikož simulovaný systém v současné
      době neexistuje, jako validní jsme model prohlásili na základě informací
      poskytnutých ve zdroji.

  \section{Rozbor tématu}

    Informace, které jsme potřebovali pro úspěšnou implementaci jsme vyhledali
    na internetu. Často posloužily jako zdroj diplomové či bakalářské práce a 
    pak také již výše zmíněný zdroj. Získané hodnoty, které jsme použili jsou vypsány níže:


  \section{Závěr}



  \section{Reference}

    \begin{enumerate}[label={[\arabic*]}]
      \item source \href{source}{source.cz}
    \end{enumerate}

  \appendix
  \newpage

  \section{Přílohy}



\end{document}
